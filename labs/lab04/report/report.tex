% Options for packages loaded elsewhere
% Options for packages loaded elsewhere
\PassOptionsToPackage{unicode}{hyperref}
\PassOptionsToPackage{hyphens}{url}
%
\documentclass[
  english,
  russian,
  12pt,
  a4paper,
  DIV=11,
  numbers=noendperiod]{scrreprt}
\usepackage{xcolor}
\usepackage{amsmath,amssymb}
\setcounter{secnumdepth}{5}
\usepackage{iftex}
\ifPDFTeX
  \usepackage[T1]{fontenc}
  \usepackage[utf8]{inputenc}
  \usepackage{textcomp} % provide euro and other symbols
\else % if luatex or xetex
  \usepackage{unicode-math} % this also loads fontspec
  \defaultfontfeatures{Scale=MatchLowercase}
  \defaultfontfeatures[\rmfamily]{Ligatures=TeX,Scale=1}
\fi
\usepackage{lmodern}
\ifPDFTeX\else
  % xetex/luatex font selection
\fi
% Use upquote if available, for straight quotes in verbatim environments
\IfFileExists{upquote.sty}{\usepackage{upquote}}{}
\IfFileExists{microtype.sty}{% use microtype if available
  \usepackage[]{microtype}
  \UseMicrotypeSet[protrusion]{basicmath} % disable protrusion for tt fonts
}{}
\usepackage{setspace}
% Make \paragraph and \subparagraph free-standing
\makeatletter
\ifx\paragraph\undefined\else
  \let\oldparagraph\paragraph
  \renewcommand{\paragraph}{
    \@ifstar
      \xxxParagraphStar
      \xxxParagraphNoStar
  }
  \newcommand{\xxxParagraphStar}[1]{\oldparagraph*{#1}\mbox{}}
  \newcommand{\xxxParagraphNoStar}[1]{\oldparagraph{#1}\mbox{}}
\fi
\ifx\subparagraph\undefined\else
  \let\oldsubparagraph\subparagraph
  \renewcommand{\subparagraph}{
    \@ifstar
      \xxxSubParagraphStar
      \xxxSubParagraphNoStar
  }
  \newcommand{\xxxSubParagraphStar}[1]{\oldsubparagraph*{#1}\mbox{}}
  \newcommand{\xxxSubParagraphNoStar}[1]{\oldsubparagraph{#1}\mbox{}}
\fi
\makeatother


\usepackage{longtable,booktabs,array}
\usepackage{calc} % for calculating minipage widths
% Correct order of tables after \paragraph or \subparagraph
\usepackage{etoolbox}
\makeatletter
\patchcmd\longtable{\par}{\if@noskipsec\mbox{}\fi\par}{}{}
\makeatother
% Allow footnotes in longtable head/foot
\IfFileExists{footnotehyper.sty}{\usepackage{footnotehyper}}{\usepackage{footnote}}
\makesavenoteenv{longtable}
\usepackage{graphicx}
\makeatletter
\newsavebox\pandoc@box
\newcommand*\pandocbounded[1]{% scales image to fit in text height/width
  \sbox\pandoc@box{#1}%
  \Gscale@div\@tempa{\textheight}{\dimexpr\ht\pandoc@box+\dp\pandoc@box\relax}%
  \Gscale@div\@tempb{\linewidth}{\wd\pandoc@box}%
  \ifdim\@tempb\p@<\@tempa\p@\let\@tempa\@tempb\fi% select the smaller of both
  \ifdim\@tempa\p@<\p@\scalebox{\@tempa}{\usebox\pandoc@box}%
  \else\usebox{\pandoc@box}%
  \fi%
}
% Set default figure placement to htbp
\def\fps@figure{htbp}
\makeatother



\ifLuaTeX
\usepackage[bidi=basic,provide=*]{babel}
\else
\usepackage[bidi=default,provide=*]{babel}
\fi
% get rid of language-specific shorthands (see #6817):
\let\LanguageShortHands\languageshorthands
\def\languageshorthands#1{}


\setlength{\emergencystretch}{3em} % prevent overfull lines

\providecommand{\tightlist}{%
  \setlength{\itemsep}{0pt}\setlength{\parskip}{0pt}}



 
\usepackage[backend=biber,langhook=extras,autolang=other*]{biblatex}
\addbibresource{bib/cite.bib}

\usepackage[]{csquotes}

\usepackage{indentfirst}
\usepackage{float}
\floatplacement{figure}{H}
\IfFileExists{plex-otf.sty}{
  %% Full TeXlive
  \IfPackageAtLeastTF{plex-otf.sty}{2024-12-06}{
    \usepackage[math,RM={Scale=0.94},SS={Scale=0.94},SScon={Scale=0.94},TT={Scale=MatchLowercase,FakeStretch=0.9},DefaultFeatures={Ligatures=Common}]{plex-otf}
  }{
    \usepackage[RM={Scale=0.94},SS={Scale=0.94},SScon={Scale=0.94},TT={Scale=MatchLowercase,FakeStretch=0.9},DefaultFeatures={Ligatures=Common}]{plex-otf}
  }
}{
  %% TinyTeX
  \usepackage{libertine}
}
\KOMAoption{captions}{tableheading}
\makeatletter
\@ifpackageloaded{caption}{}{\usepackage{caption}}
\AtBeginDocument{%
\ifdefined\contentsname
  \renewcommand*\contentsname{Содержание}
\else
  \newcommand\contentsname{Содержание}
\fi
\ifdefined\listfigurename
  \renewcommand*\listfigurename{Список иллюстраций}
\else
  \newcommand\listfigurename{Список иллюстраций}
\fi
\ifdefined\listtablename
  \renewcommand*\listtablename{Список таблиц}
\else
  \newcommand\listtablename{Список таблиц}
\fi
\ifdefined\figurename
  \renewcommand*\figurename{Рисунок}
\else
  \newcommand\figurename{Рисунок}
\fi
\ifdefined\tablename
  \renewcommand*\tablename{Таблица}
\else
  \newcommand\tablename{Таблица}
\fi
}
\@ifpackageloaded{float}{}{\usepackage{float}}
\floatstyle{ruled}
\@ifundefined{c@chapter}{\newfloat{codelisting}{h}{lop}}{\newfloat{codelisting}{h}{lop}[chapter]}
\floatname{codelisting}{Список}
\newcommand*\listoflistings{\listof{codelisting}{Листинги}}
\makeatother
\makeatletter
\makeatother
\makeatletter
\@ifpackageloaded{caption}{}{\usepackage{caption}}
\@ifpackageloaded{subcaption}{}{\usepackage{subcaption}}
\makeatother
\usepackage{bookmark}
\IfFileExists{xurl.sty}{\usepackage{xurl}}{} % add URL line breaks if available
\urlstyle{same}
\hypersetup{
  pdflang={ru-RU},
  hidelinks,
  pdfcreator={LaTeX via pandoc}}


\author{}
\date{}
\begin{document}

\renewcommand*\contentsname{Содержание}
{
\setcounter{tocdepth}{1}
\tableofcontents
}
\listoffigures
\listoftables

\setstretch{1.5}
\chapter{Лабораторная работа
№4}\label{ux43bux430ux431ux43eux440ux430ux442ux43eux440ux43dux430ux44f-ux440ux430ux431ux43eux442ux430-4}

\textbf{Студент}: Добрынин Никита \textbf{Группа}: НБИбд-01-25

\begin{center}\rule{0.5\linewidth}{0.5pt}\end{center}

\begin{enumerate}
\def\labelenumi{\arabic{enumi}.}
\item
  Цель работы Освоение процедуры компиляции и сборки программ,
  написанных на ассемблере NASM.
\item
  Самостоятельная работа

  \begin{enumerate}
  \def\labelenumii{\arabic{enumii}.}
  \tightlist
  \item
    В каталоге \textasciitilde/work/arch-pc/lab04 с помощью команды cp
    создайте копию файла hello.asm с именем lab4.asm
  \item
    С помощью любого текстового редактора внесите изменения в текст
    программы в файле lab4.asm так, чтобы вместо Hello world! на экран
    выводилась строка с вашими фамилией и именем.
  \item
    Оттранслируйте полученный текст программы lab4.asm в объектный файл.
    Выполните компоновку объектного файла и запустите получившийся
    исполняемый файл.
  \item
    Скопируйте файлы hello.asm и lab4.asm в Ваш локальный репозиторий в
    ката- лог \textasciitilde/work/study/2023-2024/\enquote{Архитектура
    компьютера}/arch-pc/labs/lab04/. Загрузите файлы на Github.
  \end{enumerate}
\end{enumerate}

\begin{center}\rule{0.5\linewidth}{0.5pt}\end{center}

\begin{enumerate}
\def\labelenumi{\arabic{enumi}.}
\setcounter{enumi}{2}
\tightlist
\item
  Выполнение работы
\end{enumerate}

\pandocbounded{\includegraphics[keepaspectratio]{Home/Pictures/Screenshots/Pic1.jpg}}
Рис.1

Создал каталог, перешел в него, создал текстовый файл hello.asm и окрыл
его с помощью gedit (Рис.1)

\pandocbounded{\includegraphics[keepaspectratio]{Home/Pictures/Screenshots/Pic2.jpg}}
Рис.2

Ввел показанный текст (Рис.2)

\begin{figure}[H]

{\centering \pandocbounded{\includegraphics[keepaspectratio]{Home/Pictures/Screenshots/Pic3.jpg}}

}

\caption{Рис.3}

\end{figure}%

Провел компиляцию текста программы при помощи команды nasm -f elf
hello.asm (Рис.3)

\begin{figure}[H]

{\centering \pandocbounded{\includegraphics[keepaspectratio]{Home/Pictures/Screenshots/Pic4.jpg}}

}

\caption{Рис.4}

\end{figure}%

Выполнил полный вариант командной строки NASM (Рис.4)

\begin{figure}[H]

{\centering \pandocbounded{\includegraphics[keepaspectratio]{Home/Pictures/Screenshots/Pic5.jpg}}

}

\caption{Рис.5}

\end{figure}%

Передал объектный файл на обработку компановщика (Рис.5)

\begin{figure}[H]

{\centering \pandocbounded{\includegraphics[keepaspectratio]{Home/Pictures/Screenshots/Pic6.jpg}}

}

\caption{Рис.6}

\end{figure}%

Задал имя исполняемому файлу (Рис.6)

\begin{figure}[H]

{\centering \pandocbounded{\includegraphics[keepaspectratio]{Home/Pictures/Screenshots/Pic7.jpg}}

}

\caption{Рис.7}

\end{figure}%

Запустил исполняемый файл (Рис.7)

\begin{center}\rule{0.5\linewidth}{0.5pt}\end{center}

\#Выполнение самостоятельной работы

\begin{figure}[H]

{\centering \pandocbounded{\includegraphics[keepaspectratio]{Home/Pictures/Screenshots/Pic8.jpg}}

}

\caption{Рис.8}

\end{figure}%

Создал копию файла hello.asm с именем lab4.asm (Рис.8)

\begin{figure}[H]

{\centering \pandocbounded{\includegraphics[keepaspectratio]{Home/Pictures/Screenshots/Pic11.jpg}}

}

\caption{Рис.9}

\end{figure}%

Внес изменения в текст программы что бы она выводила мои имя и фамилию
(Рис.9)

\begin{figure}[H]

{\centering \pandocbounded{\includegraphics[keepaspectratio]{Home/Pictures/Screenshots/Pic8.jpg}}

}

\caption{Рис.10}

\end{figure}%

Оттранслировал полученный текст программы в объектный файл (Рис.10)

\begin{figure}[H]

{\centering \pandocbounded{\includegraphics[keepaspectratio]{Home/Pictures/Screenshots/Pic9.jpg}}

}

\caption{Рис.11}

\end{figure}%

Выполнил компановку объектного файла

\begin{figure}[H]

{\centering \pandocbounded{\includegraphics[keepaspectratio]{Home/Pictures/Screenshots/Pic10.jpg}}

}

\caption{Рис.12}

\end{figure}%

Запустил файл, всё сработало корректно


\printbibliography



\end{document}
